\section{Jacobi Plate Temperature Simulation}
\paragraph{Goals and algorithms.}
For the two programs implemented to simulate a square plate's temperature
distribution over time until a stable heat was achieved throughout the
entire plate. The jacobi method was chosen as the algorithm to simulate this.
Parallelization of the algorithm was done by partitioning the plate into
sectioned rows that each thread would perform the calculation on exclusively
from one another.

\paragraph{Jacobi Implemented in C++.}
There are three implementations of Jacobi in C++: A Serial version, and two
parallel versions. This was done because the initial parallel version, which
created new threads each iteration, performed poorly compared to the serial
version; The second parallel version instead created the child threads once,
and used semaphores to synchronize the threads each iteration instead. This
second parallel version's performance was much closer to the serial version
when utilizing only a single child thread. It also proved to scale with
additional threads at a rate much better than the first parallel version,
eclipsing both the performance of the serial and first parallel version.

\par Each version of jacobi was timed using the same rules. The clock starts
after arguments to the program are checked, and just before the arrays
simulating the plate are initialized. This timer ends after the total number
of cells below the value 50.0 is calculated.

\paragraph{Jacobi Implemented in Rust.}
Jacobi had three versions implemented in Rust as well, for performance anomalies
just like the C++ implementation. The three versions are as follows: A Serial
version, a stdlib parallel version, and a crossbeam parallel version. The serial
version of jacobi was a straightforward port from the C++ one to Rust, while
the crossbeam version was made due to the stdlib of Rust being insufficient for
implementing concurrent jacobi.

\par The first major change made in all versions was
from allocating the array that represents the plate as a static global to
dynamically allocating it during runtime as a vector out of necessity. This was
done as Rust does not allow for global mutable variables, and stack allocation
of the plate would overflow the stack. The second import implementation detail
difference for the stdlib version was the huge costs incurred in-order to
satisfy Rust's borrow checker without the aid of the recently removed
{\cf std::scoped::thread}. The crossbeam version of rust remedies the
performance-costing changes made to accommodate the Rust stdlib by utilizing a
third-party library written in Rust that emulates the functionality provided by
{\cf std::scoped::thread}.

%TODO(todd): Write more

\paragraph{User Experience writing Jacobi in C++.}
This was very simple to implement, it was based off of a single threaded C
version so most of the math parts were already figured out, it was just a
matter of making it concurrent. The only major issue that arose was a small
thread creation loop that went unnoticed until benchmarking was done, it
spawned threads but then never used them which caused some performance
anomalies until it was found and removed.

\paragraph{User Experience writing Jacobi in Rust.}
The serial version of jacobi implemented in Rust was very straightforward. Rust
lends itself well to porting code from C++ to Rust, as the Rust version was
based of the C++ version. Rust provides the convenience of zero-initializing
every variable by default, which removed the need for an amount of the jacobi
initialization code\footnote{This can also be done explicitly in C++}.
Attempting to allocate the array that represented the plate managed to trigger
a segfault, although a minor one, in Rust for the first time. It appears that
Rust cannot guarantee against over-allocating the stack understandably.

\par Implementing the first parallel version using only the standard library
was very difficult, as sharing portions of an array among threads and
guaranteeing the lifetimes of the grid portions is necessary. It was a battle
against Rust's borrow checker rather than solving the problem, as even though
joining the threads at the end should guarantee that they do not outlive the
main thread, Rust was unable to understand this and failed to compile the program.
%TODO(todd): finish
