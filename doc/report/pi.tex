\chapter{Pi Digits Computation}
\paragraph{Goals and algorithms.}
The goal of these program implementations were to compute the number pi using
Riemann sums. In other words combining the area under the curve split into small
rectangular intervals. For
parallelization, the strategy used was to evenly distribute the
number interval areas to be calculated among the threads created by the
program, and then join the results of each thread at the end to form the final
answer for pi. To allow for a substantial enough runtime to measure, the
programs were run using an absurdly high number of intervals.

\paragraph{Pi Implemented in C++.}
The C++ implementation used the POSIX pthread library, and creates $n$
threads, as the user specifies, to compute the area under the curve. In order
to isolate the performance of the thread creation, deletion, and parallelization
of pthreads, the timer used for measuring the program's runtime is started after
all initialization of variables occurs, and just before creation of the child
threads.

Once all
threads are created, the main thread waits idle while trying to join all child
threads. As each child thread terminates, the final sum of pi is added to. Once
all threads are finished, the final result is calculated and the timer stops.

\paragraph{Pi Implemented in Rust.}
The Rust implementation, like the C++ implementation, takes $n$ threads, as the
user specifies, to compute the area under the curve. Another similarity the Rust
version shares with its C++ counterpart is that it's timer begins after the all
initialization of variables, and just before creation of the child threads.

Once each thread finishes its work, it sends it's solution to a channel located
on the main thread where it is added to a global sum. Once all threads are
finished, the sum final sum is calculated and the timer stops.

\paragraph{User Experience writing pi in C++.}
Overall this program was trivial to implement in C++, as the problem is not very
complex, and room for memory errors to appear is narrow. There were no recorded
memory errors encountered during the writing of the C++ program, and no major
usability faults regarding pthreads, with one exception: Passing arguments to
a child thread is certainly type unsafe, and memory unsafe, and could be
a potential culprit for many runtime errors if not carefully managed. This is
due to C++ pthreads requiring a single {\cf void *} for all arguments, which
guarantees almost nothing about the data being passed into the thread, and
cannot allow for the compiler to check for safety. In the case of Pi, this was
not a major issue. Only a single integer value was to be passed into each
thread upon its creation, and so no risk for accessing out-of-bounds memory was
present.

\paragraph{User Experience writing pi in Rust.}
Writing this program felt much simpler and more inutitve than C++ with pthreads.
Since channels make asynchronous sums very trivial, the hard part was pretty
much already solved from the begining. The implementation looks nearly identical
to what a single threaded pi solver implementation would look like, except for
the addition of a thread spawn occuring in the main loop and a channel being
used to collect the sums. Even an individual with little knowledge of the
language could easily figure out whats going on, a feature that isn't shared
with the C++ implementation.
