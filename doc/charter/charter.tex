\documentclass[10pt,a4paper]{report}
\usepackage[top=1in, left=0.5in, right=0.5in, bottom=1.5in]{geometry}
\usepackage{titlesec}
\usepackage{tabulary}
\usepackage{tabularx}

\pagenumbering{gobble}
\begin{document}
% Title
\begin{center}
	{\Large Project Charter: {\bfseries Rust Evaluation}}
	- {\large{\itshape Ethan Larkham} and {\itshape Todd Gaunt}}
	- {\large October 13, 2017}
\end{center}
\vspace{0.5em}
\paragraph{Background}
Rust is a new programming language being developed by Mozilla, released in 2015
aimed at system application developers. Many of its competitors such as C++ are
not memory safe and are prone to security problems such buffer overflow
exploits. Rust aims to resolve these issues as a completely new language by
enforcing a strict emphasis on memory safety and program correctness during
compile time rather than run time, while maintaining competitive performance.
The goal of Rust to reduce security problems while maintaining comparable
performance with C++ in concurrent programs will be the primary subject of this
evaluation.
\paragraph{Goals}
\begin{flushleft}
	\begin{itemize}
		\item Measure performance of concurrent programs
		\item Analysis of Rust compile time error checking
		\item Evaluation of language ergonomics
	\end{itemize}
\end{flushleft}
\paragraph{Scope}
The end result of the project will be a research paper and poster of the
project's outcome.  We will be implementing 4 program specifications twice, once in C++ and
once in Rust. This will allow for analysis of the differences in computation
speed, memory usage, start-up times, processor utilization, and ease of
development for each language.
At this time we intend to implement a concurrent hash table (which will be used to count the
amount of words in text files), Jacobi temperature simulation,  Pi
digits, and the Gaussian function. These programs were chosen to be implemented
as they can all be parallelized. Each program will be implemented to the same
specification in both C++ and Rust; These languages will be alternated between
team members to allow for both team members to evaluate each language.
\paragraph{Key Stakeholders}
\begin{flushleft}
	\begin{tabulary}{\textwidth}{ | L | L | }
		\hline
		Client/Sponsor & Professor Philip J. Hatcher \\
		\hline
		Project Manager & Collette Matthias Powers \\
		\hline
		Project Team Members & Ethan Larkham and Todd Gaunt \\
		\hline
	\end{tabulary}
\end{flushleft}
\paragraph{Project Milestones}
\begin{flushleft}
	\begin{tabulary}{\textwidth}{ | L | L | }
		\hline
		Learn Rust Language & By End of October, 2017 \\
		\hline
		Implement and Analyze performance of Pi Digits & By Mid November, 2017 \\
		\hline
		Implement and Analyze performance of Jacobi Temperature
		Simulation & By end of November, 2017 \\
		\hline
		Implement and Analyze performance of Generic concurrent Hash Table
		& By end of December, 2017 \\
		\hline
		Implement and Analyze performance of Gaussian Function &
		By end of January, 2018 \\
		\hline
		Analysis of generated assembly code of programs &
		Before end of year, 2018 \\
		\hline
		Compile Findings into Research Paper & By end of year, 2018 \\
		\hline
	\end{tabulary}
\end{flushleft}
\paragraph{Constraints, Assumptions, Risks and Dependencies}
\begin{flushleft}
	\begin{tabulary}{\textwidth}{ | L | L | }
		\hline
		Constraints & Unable to objectively quantify language
		ergonomics. \\
		\hline
		Assumptions & Rust will be able to implement programs in a
		comparable way to C++. There is an appropriate platform to run
		the programs on. \\
		\hline
		Risks and Dependencies & We need to find tools to test and
    benchmark Rust with and learn the design patterns commonly
    used by the community. Tools for analyzing Rust may not work as expected. \\
		\hline
	\end{tabulary}
\end{flushleft}
\paragraph{Approval Signatures}
\begin{flushleft}
	\begin{tabularx}{\textwidth}{@{\extracolsep{\fill}} | X | X | }
		\hline
		& \\
		& \\
		\hline
		Philip J. Hatcher, Project Client/Sponsor & Collette Matthias
		Powers, Project Manager \\
		\hline
	\end{tabularx}
\end{flushleft}
\end{document}
