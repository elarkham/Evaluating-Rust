\documentclass[10pt,a4paper]{report}
\usepackage[top=1in, left=0.5in, right=0.5in, bottom=1.5in]{geometry}
\usepackage{titlesec}
\usepackage{tabulary}
\usepackage{tabularx}

\pagenumbering{gobble}
\begin{document}
% Title
\begin{center}
	{\Large Project Charter: {\bfseries Rust Evaluation}}
	- {\large{\itshape Ethan Larkham} and {\itshape Todd Gaunt}}
	- {\large October 13, 2017}
\end{center}
\vspace{0.5em}
\paragraph{Background}
Rust is a new programming language being developed by Mozilla, released in 2015
aimed at system application developers. Many of its competitors such as C++ are
not memory safe and are prone to security problems such buffer overflow
exploits. Rust aims to resolve these issues as a completely new language by
enforcing a strict emphasis on memory safety and program correctness during
compile time rather than run time, while maintaining competitive performance.
The goal of Rust to reduce security problems while maintaining comparable
performance with C++ in concurrent programs will be the primary subject of this
evaluation.
\paragraph{Goals}
\begin{flushleft}
	\begin{itemize}
		\item Measure performance of concurrent programs
		\item Analysis of Rust compile time error checking
		\item Evaluation of language ergonomics
	\end{itemize}
\end{flushleft}
\paragraph{Scope}
The end result of the project will be a presentation, most likely in the form
of a research paper and or poster, of our findings during our evaluation of
Rust.  We will be implementing 4 program specifications twice, once in C++ and
once in Rust. This will allow for analysis of the differences in performance,
ease of use and generated assembly code of each language. At this time we
intend to implement a concurrent hash table, Jacobi temperature simulation,  Pi
digits, and the Gaussian function. Both of us will be implementing the two
programs for each specification separately and alternating between which of us
will do the Rust version and the C++ version.
\paragraph{Key Stakeholders}
\begin{flushleft}
	\begin{tabulary}{\textwidth}{ | L | L | }
		\hline
		Client/Sponser & Professor Philip J. Hatcher \\
		\hline
		Project Manager & Collette Matthias Powers \\
		\hline
		Project Team Members & Ethan Larkham and Todd Gaunt \\
		\hline
	\end{tabulary}
\end{flushleft}
\paragraph{Project Milestones}
\begin{flushleft}
	\begin{tabulary}{\textwidth}{ | L | L | }
		\hline
		Learn Rust Language & November 5, 2017 \\
		\hline
		Implement and Analyze performance of Concurrent Hash Table
		& November 19, 2017 \\
		\hline
		Implement and Analyze performance of Jacobi Temperature
		Simulation & December 3, 2017 \\
		\hline
		Implement and Analyze performance of Pi Digits & December 11,
		2017 \\
		\hline
		Implement and Analyze performance of Gaussian Function &
		February 5, 2018 \\
		\hline
		Analysis of generated assembly code of programs &
		March 5, 2018 \\
		\hline
		Compile Findings into Research Paper & April 15, 2018 \\
		\hline
	\end{tabulary}
\end{flushleft}
\paragraph{Constraints, Assumptions, Risks and Dependencies}
\begin{flushleft}
	\begin{tabulary}{\textwidth}{ | L | L | }
		\hline
		Constraints & Unable to objectively quantify language
		ergonomics \\
		\hline
		Assumptions & Rust will be able to implement programs in a
		comparable way to C++. \\
		\hline
		Risks and Dependencies & We need to find tools to test and
    benchmark Rust with and learn the design patterns commonly
    used by the community. \\
		\hline
	\end{tabulary}
\end{flushleft}
\paragraph{Approval Signatures}
\begin{flushleft}
	\begin{tabularx}{\textwidth}{@{\extracolsep{\fill}} | X | X | }
		\hline
		& \\
		& \\
		\hline
		Philip J. Hatcher, Project Client/Sponser & Collette Matthias
		Powers, Project Manager \\
		\hline
	\end{tabularx}
\end{flushleft}
\end{document}
